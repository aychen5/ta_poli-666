\documentclass[ignorenonframetext,]{beamer}
\setbeamertemplate{caption}[numbered]
\setbeamertemplate{caption label separator}{: }
\setbeamercolor{caption name}{fg=normal text.fg}
\beamertemplatenavigationsymbolsempty
\usepackage{lmodern}
\usepackage{amssymb,amsmath}
\usepackage{ifxetex,ifluatex}
\usepackage{fixltx2e} % provides \textsubscript
\ifnum 0\ifxetex 1\fi\ifluatex 1\fi=0 % if pdftex
  \usepackage[T1]{fontenc}
  \usepackage[utf8]{inputenc}
\else % if luatex or xelatex
  \ifxetex
    \usepackage{mathspec}
  \else
    \usepackage{fontspec}
  \fi
  \defaultfontfeatures{Ligatures=TeX,Scale=MatchLowercase}
\fi
\usetheme[]{AnnArbor}
\usecolortheme{dolphin}
\usefonttheme{structuresmallcapsserif}
% use upquote if available, for straight quotes in verbatim environments
\IfFileExists{upquote.sty}{\usepackage{upquote}}{}
% use microtype if available
\IfFileExists{microtype.sty}{%
\usepackage{microtype}
\UseMicrotypeSet[protrusion]{basicmath} % disable protrusion for tt fonts
}{}
\newif\ifbibliography
\hypersetup{
            pdftitle={Instrumental Variables},
            pdfauthor={Annie Chen},
            pdfborder={0 0 0},
            breaklinks=true}
\urlstyle{same}  % don't use monospace font for urls
\usepackage{color}
\usepackage{fancyvrb}
\newcommand{\VerbBar}{|}
\newcommand{\VERB}{\Verb[commandchars=\\\{\}]}
\DefineVerbatimEnvironment{Highlighting}{Verbatim}{commandchars=\\\{\}}
% Add ',fontsize=\small' for more characters per line
\usepackage{framed}
\definecolor{shadecolor}{RGB}{248,248,248}
\newenvironment{Shaded}{\begin{snugshade}}{\end{snugshade}}
\newcommand{\KeywordTok}[1]{\textcolor[rgb]{0.13,0.29,0.53}{\textbf{#1}}}
\newcommand{\DataTypeTok}[1]{\textcolor[rgb]{0.13,0.29,0.53}{#1}}
\newcommand{\DecValTok}[1]{\textcolor[rgb]{0.00,0.00,0.81}{#1}}
\newcommand{\BaseNTok}[1]{\textcolor[rgb]{0.00,0.00,0.81}{#1}}
\newcommand{\FloatTok}[1]{\textcolor[rgb]{0.00,0.00,0.81}{#1}}
\newcommand{\ConstantTok}[1]{\textcolor[rgb]{0.00,0.00,0.00}{#1}}
\newcommand{\CharTok}[1]{\textcolor[rgb]{0.31,0.60,0.02}{#1}}
\newcommand{\SpecialCharTok}[1]{\textcolor[rgb]{0.00,0.00,0.00}{#1}}
\newcommand{\StringTok}[1]{\textcolor[rgb]{0.31,0.60,0.02}{#1}}
\newcommand{\VerbatimStringTok}[1]{\textcolor[rgb]{0.31,0.60,0.02}{#1}}
\newcommand{\SpecialStringTok}[1]{\textcolor[rgb]{0.31,0.60,0.02}{#1}}
\newcommand{\ImportTok}[1]{#1}
\newcommand{\CommentTok}[1]{\textcolor[rgb]{0.56,0.35,0.01}{\textit{#1}}}
\newcommand{\DocumentationTok}[1]{\textcolor[rgb]{0.56,0.35,0.01}{\textbf{\textit{#1}}}}
\newcommand{\AnnotationTok}[1]{\textcolor[rgb]{0.56,0.35,0.01}{\textbf{\textit{#1}}}}
\newcommand{\CommentVarTok}[1]{\textcolor[rgb]{0.56,0.35,0.01}{\textbf{\textit{#1}}}}
\newcommand{\OtherTok}[1]{\textcolor[rgb]{0.56,0.35,0.01}{#1}}
\newcommand{\FunctionTok}[1]{\textcolor[rgb]{0.00,0.00,0.00}{#1}}
\newcommand{\VariableTok}[1]{\textcolor[rgb]{0.00,0.00,0.00}{#1}}
\newcommand{\ControlFlowTok}[1]{\textcolor[rgb]{0.13,0.29,0.53}{\textbf{#1}}}
\newcommand{\OperatorTok}[1]{\textcolor[rgb]{0.81,0.36,0.00}{\textbf{#1}}}
\newcommand{\BuiltInTok}[1]{#1}
\newcommand{\ExtensionTok}[1]{#1}
\newcommand{\PreprocessorTok}[1]{\textcolor[rgb]{0.56,0.35,0.01}{\textit{#1}}}
\newcommand{\AttributeTok}[1]{\textcolor[rgb]{0.77,0.63,0.00}{#1}}
\newcommand{\RegionMarkerTok}[1]{#1}
\newcommand{\InformationTok}[1]{\textcolor[rgb]{0.56,0.35,0.01}{\textbf{\textit{#1}}}}
\newcommand{\WarningTok}[1]{\textcolor[rgb]{0.56,0.35,0.01}{\textbf{\textit{#1}}}}
\newcommand{\AlertTok}[1]{\textcolor[rgb]{0.94,0.16,0.16}{#1}}
\newcommand{\ErrorTok}[1]{\textcolor[rgb]{0.64,0.00,0.00}{\textbf{#1}}}
\newcommand{\NormalTok}[1]{#1}

% Prevent slide breaks in the middle of a paragraph:
\widowpenalties 1 10000
\raggedbottom

\AtBeginPart{
  \let\insertpartnumber\relax
  \let\partname\relax
  \frame{\partpage}
}
\AtBeginSection{
  \ifbibliography
  \else
    \let\insertsectionnumber\relax
    \let\sectionname\relax
    \frame{\sectionpage}
  \fi
}
\AtBeginSubsection{
  \let\insertsubsectionnumber\relax
  \let\subsectionname\relax
  \frame{\subsectionpage}
}

\setlength{\parindent}{0pt}
\setlength{\parskip}{6pt plus 2pt minus 1pt}
\setlength{\emergencystretch}{3em}  % prevent overfull lines
\providecommand{\tightlist}{%
  \setlength{\itemsep}{0pt}\setlength{\parskip}{0pt}}
\setcounter{secnumdepth}{0}
\usepackage{multirow}
\usepackage{tikz}
\usetikzlibrary{positioning,shapes.geometric,arrows}
\usetikzlibrary{decorations.markings}
\usepackage{pgfplots}
\usepackage{graphicx}
\graphicspath{ {./images/} }
\newcommand{\indep}{\rotatebox[origin=c]{90}{$\models$}}

\title{Instrumental Variables}
\author{Annie Chen}
\date{February 26, 2020}

\begin{document}
\frame{\titlepage}

\begin{frame}{}

\begin{center}\includegraphics[width=0.5\linewidth]{iv_memes} \end{center}

\end{frame}

\begin{frame}{IV Assumptions}

\color{red}{Which are directly testable from data?}

\begin{itemize}[<+->]
\tightlist
\item
  Ignorability (Exogeneity)

  \begin{itemize}[<+->]
  \tightlist
  \item
    \(\{Y_i(z,d),D_i(1),D_i(0)\}\indep Z_i\)
  \end{itemize}
\item
  SUTVA

  \begin{itemize}[<+->]
  \tightlist
  \item
    \(D(z) = D(z^{\prime})\) if \(z = z^{\prime}\)
  \item
    \(Y(z, d) = Y(z^{\prime}, d^{\prime})\)
  \end{itemize}
\item
  Exclusion Restriction

  \begin{itemize}[<+->]
  \tightlist
  \item
    \(Y_i(z = 1,d) = Y_i(z = 0,d)\) for \(d = 0,1\) 
  \end{itemize}
\item
  Relevance

  \begin{itemize}[<+->]
  \tightlist
  \item
    \(Cov(D_i, Z_i) \neq 0\)
  \end{itemize}
\item
  Monotonicity

  \begin{itemize}[<+->]
  \tightlist
  \item
    \(D_i(1) \geq D_i(0)\) for all \(i\)
  \end{itemize}
\end{itemize}

\end{frame}

\begin{frame}{Valid Instruments}

\begin{itemize}[<+->]
\tightlist
\item
  Let's play a game:
  \color{red}{is $Z$ a valid instrument that identifies a causal effect of $D$ on $Y$?}
\end{itemize}

\begin{center}
\begin{tikzpicture}[node distance=5cm,auto,>=latex', scale =1.5, transform shape]
     \node[text centered] (d) {$D$};
     \node[right = 2 of d, text centered] (y) {$Y$};
      \node[above left = 1 of d, text centered] (z) {$Z$};
        \node[below right = 1 of d, text centered] (e) {$U$};   
      \draw[->, line width= 1] (d) --(y);
      \draw[->, line width= 1, bend left, dashed] (e) --(d);
      \draw[->, line width= 1, bend left, dashed] (e) --(y);
      \draw[->, line width= 1] (z) --(d);
      \draw[->, line width= 1] (z) --(y);
\end{tikzpicture}
\end{center}

\end{frame}

\begin{frame}{Valid Instruments}

\begin{itemize}[<+->]
\tightlist
\item
  \color{red}{Can we test the exclusion restriction by checking the association between Z and Y after conditioning on D?}
\end{itemize}

\begin{center}
\begin{tikzpicture}[node distance=5cm,auto,>=latex', scale =1.5, transform shape]
     \node[rectangle, draw] (d) {$D$};
     \node[right = 2 of d, text centered] (y) {$Y$};
      \node[above left = 1 of d, text centered] (z) {$Z$};
        \node[below right = 1 of d, text centered] (e) {$U$};   
      \draw[->, line width= 1] (d) --(y);
      \draw[->, line width= 1, bend left, dashed] (e) --(d);
      \draw[->, line width= 1, bend left, dashed] (e) --(y);
      \draw[->, line width= 1] (z) --(d);
\end{tikzpicture}
\end{center}

\end{frame}

\begin{frame}{Valid Instruments}

\begin{itemize}[<+->]
\tightlist
\item
  and here? Suppose the instrument \(Z\) is unobserved, but we have a
  proxy \(V\).
\end{itemize}

\begin{center}
\begin{tikzpicture}[node distance=5cm,auto,>=latex', scale =1.5, transform shape]
     \node[text centered] (d) {$D$};
     \node[right = 2 of d, text centered] (y) {$Y$};
        \node[above left = 1 of d, text centered] (z) {$Z$};
      \node[left = 1 of z, text centered] (v) {$V$};    
        \node[below right = 1 of d, text centered] (e) {$U$};   
      \draw[->, line width= 1] (d) --(y);
      \draw[->, line width= 1, bend left] (z) --(d);
      \draw[->, line width= 1] (z) --(v);
      \draw[->, line width= 1, bend left, dashed] (e) --(d);
      \draw[->, line width= 1, bend left, dashed] (e) --(y);
\end{tikzpicture}
\end{center}

\end{frame}

\begin{frame}{Valid Instruments}

\begin{itemize}[<+->]
\tightlist
\item
  Can we make this one work?
\end{itemize}

\begin{center}
\begin{tikzpicture}[node distance=5cm,auto,>=latex', scale =1.5, transform shape]
     \node[text centered] (d) {$D$};
     \node[right = 2 of d, text centered] (y) {$Y$};
      \node[above left = 1 of d, text centered] (z) {$Z$};
        \node[below right = 1 of d, text centered] (e) {$U$};   
        \node[above left = 1 of y, text centered] (x) {$X$};    
      \draw[->, line width= 1] (d) --(y);
      \draw[->, line width= 1] (x) --(z);
      \draw[->, line width= 1] (x) --(y);
      \draw[->, line width= 1, bend left, dashed] (e) --(d);
      \draw[->, line width= 1, bend left, dashed] (e) --(y);
      \draw[->, line width= 1] (z) --(d);
\end{tikzpicture}
\end{center}

\end{frame}

\begin{frame}{Valid Instruments}

\begin{itemize}[<+->]
\tightlist
\item
  Can we make this one work?\footnote<.->{Note that this is now a
    conditional instrument and we adjust our assumptions accordingly.
    I.e., \(\{Y(z, d), D(1), D(0)\} \perp Z|X\)}
\end{itemize}

\begin{center}
\begin{tikzpicture}[node distance=5cm,auto,>=latex', scale =1.5, transform shape]
     \node[text centered] (d) {$D$};
     \node[right = 2 of d, text centered] (y) {$Y$};
      \node[above left = 1 of d, text centered] (z) {$Z$};
        \node[below right = 1 of d, text centered] (e) {$U$};   
        \node[above left = 1 of y, rectangle, draw, text centered] (x) {$X$};   
      \draw[->, line width= 1] (d) --(y);
      \draw[->, line width= 1] (x) --(z);
      \draw[->, line width= 1] (x) --(y);
      \draw[->, line width= 1, bend left, dashed] (e) --(d);
      \draw[->, line width= 1, bend left, dashed] (e) --(y);
      \draw[->, line width= 1] (z) --(d);
\end{tikzpicture}
\end{center}

\end{frame}

\begin{frame}{Violating IV Assumptions}

\begin{center}
    \begin{tikzpicture}[node distance=5cm,auto,>=latex', scale =1.5, transform shape]
      \node[text centered] (d) {$D$};
      \node[right = 2 of d, text centered] (y) {$Y$};
      \node[above left = 1 of d, text centered] (z) {$Z$};
      \node[below right = 1 of d, text centered] (u1) {$U$};    
      \node[right = 2.5 of z, text centered] (u2) {$U$};    
      \node[left = 2.5 of d, text centered] (u3) {$U$}; 
      \draw[->, line width= 1] (d) --(y);
      \draw[->, line width= 1, bend left, dashed] (u1) --(d);
      \draw[->, line width= 1, bend left, dashed] (u1) --(y);
      \draw[->, line width= 1] (z) --  (d) node[midway, below, sloped] {\tiny{a}} ;
      \draw[->, line width= 1] (z) --  (d) ;
      \draw[->, line width= 1, dashed] (u2) --  (z) node[midway, above, sloped]{\tiny{b}} ;
      \draw[->, line width= 1, dashed] (u2) --  (y) ;
      \draw[->, line width= 1, dashed] (u3) --  (z) node[midway, above, sloped]{\tiny{c}} ;
      \draw[->, line width= 1, dashed] (u3) --  (d) ;
      \draw[->, line width= 1, dashed] (z) --  (y) node[midway, above, sloped]{\tiny{d}} ;
    \end{tikzpicture}
\end{center}

\end{frame}

\begin{frame}{Review LATE Framework}

\begin{itemize}[<+->]
\tightlist
\item
  For unit \(i\), if \(Z_i \in \{0, 1\}\) is the instrument
  (encouragement) and \(D_i \in \{0, 1\}\) is treatment, then we have 4
  principal strata (latent class variable \emph{C}).
\end{itemize}

\begin{center}
\begin{tabular}{l|cc}
$C_i$ & \multicolumn{2}{c}{$D_i(z)$}\\
\hline
 & $Z_i = 1$ & $Z_i = 0$\\
\hline
Compliers & \color{blue}{1} & \color{blue}{0}\\
Always-takers  & \color{blue}{1} & \color{blue}{1} \\
Never-takers & \color{blue}{0} & \color{blue}{0} \\
Defiers & \color{blue}{0} & \color{blue}{1}
\end{tabular}
\end{center}

\begin{itemize}[<+->]
\tightlist
\item
  Can we determine which \emph{C} unit \(i\) falls under from our
  observed data?
\item
  Say \(Z_i = 1\) and \(D_i = 0\). What type(s) could unit \(i\) be?
\end{itemize}

\end{frame}

\begin{frame}{IV Estimand and 2SLS Estimator}

\begin{itemize}[<+->]
\tightlist
\item
  Intent-to-treat (ITT) effect: \(\mathbb{E}[Y_i(z = 1) - Y_i(z = 0)]\)
\item
  \(ITT_d\): \(\mathbb{E}[D_i(z = 1) - D_i(z = 0)]\)
\item
  Local Average Treatment Effect (LATE):\footnote<.->{also known as
    Complier Average Causal Effect (CACE) ref. GG p.144 and p.180} =
  \(\frac{ITT}{ITT_d}\) 
\item
  Can be estimated with 2SLS assuming monotonicity and excludability.
\end{itemize}

\end{frame}

\begin{frame}{Weak Instruments}

\begin{itemize}[<+->]
\tightlist
\item
  If instrument(s) are so weak that there is no first stage, the 2SLS
  distribution is centered on the plim of OLS.
\item
  Run a joint-significance test (f-test) under the null that your
  instrument(s) is/are weak.
\item
  The \emph{rule of thumb} given by Staiger and Stock (1997) is a
  first-stage F-statistic \textgreater{} 10 (in which case, we may
  reject the null and conclude the instrument is
  relevant).\footnote<.->{Stock and Yogo (2005) tabulate the critical
    values needed to reject null for some tolerable level of relative
    bias (between OLS and 2SLS estimator).}
\end{itemize}

\end{frame}

\begin{frame}{Under Identification}

\begin{itemize}[<+->]
\tightlist
\item
  Consider this regression model:
  \[Y = \beta_0  + \beta_1X + \beta_2W+ u\]

  \begin{itemize}[<+->]
  \tightlist
  \item
    \(X\) is your endogenous regressor
  \item
    \(W\) is an exogenous covariate
  \item
    You need one instrument, \(Z\), to \emph{just-}idenitify the IV
    model.
  \end{itemize}
\item
  Now suppose you have \textgreater{}1 endogenous regressors:
  \[Y = \beta_0  + \beta_1X_1 + \beta_2X_2 + ... \beta_kX_k + \gamma W+ u\]

  \begin{itemize}[<+->]
  \tightlist
  \item
    \(X_1, X_2, ..., X_k\) are your \(k\) endogenous regressors
  \item
    And you have \(Z_1, ..., Z_m\) instruments
  \item
    Where \(m < k\) = underidenfication, \(m > k\) = overid, \(m=k\) is
    just-identified.
  \end{itemize}
\end{itemize}

\end{frame}

\begin{frame}{Over Identification}

\begin{itemize}[<+->]
\tightlist
\item
  When you have more instruments than endogenous regressors, you can
  conduct an overidentification test.
\item
  Sargan-Hansen Test (\(\mathcal{J}\)-test): the test statistic is
  \(\mathcal{J}= mF\) where \(F\) is the F-statistic for the null that
  coefficients are jointly zero in
  \(\hat{u} = \beta_0 + \beta_1Z_1 + ... + \beta_mZ_m + \gamma W + \nu\).
\item
  \(\mathcal{J} \sim \mathcal{X}^2_{m-k}\) (\(k\) is number of
  endogenous regressors, \(m\) is number of instruments)
\item
  Null hypothesis is that all instruments are uncorrelated with
  \(\hat{u}\)
\item
  Assumes homoskedasticity\footnote<.->{if heteroskedastic, see
    Kleibergen--Paap rk Wald statistic}
\item
  \color{red}{Is this really a test for exogeneity?}
\end{itemize}

\end{frame}

\begin{frame}[fragile]{\texttt{AER::ivreg()}}

\begin{Shaded}
\begin{Highlighting}[]
\NormalTok{ivmod <-}\StringTok{ }\KeywordTok{ivreg}\NormalTok{(y }\OperatorTok{~}\StringTok{ }\NormalTok{x1 }\OperatorTok{+}\StringTok{ }\NormalTok{w1 }\OperatorTok{|}\StringTok{ }\NormalTok{w1 }\OperatorTok{+}\StringTok{ }\NormalTok{z1 }\OperatorTok{+}\StringTok{ }\NormalTok{z2, data)}
\KeywordTok{summary}\NormalTok{(ivmod, }\DataTypeTok{diagnostics =} \OtherTok{TRUE}\NormalTok{)}\OperatorTok{$}\NormalTok{diagnostics}
\end{Highlighting}
\end{Shaded}

\begin{verbatim}
##                  df1 df2  statistic      p-value
## Weak instruments   2  60 17.4114496 1.089283e-06
## Wu-Hausman         1  60 34.3521510 2.093826e-07
## Sargan             1  NA  0.1457748 7.026062e-01
\end{verbatim}

\end{frame}

\section{ADDITIONAL NOTES}\label{additional-notes}

\begin{frame}{Limited Information Maximum Likelihood (LIML)}

\begin{itemize}[<+->]
\tightlist
\item
  Alternative to 2SLS is LIML estimator, which has better small sample
  properties than 2SLS
\end{itemize}

\end{frame}

\end{document}
